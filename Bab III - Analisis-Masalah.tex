% ============================================================================================
% BAB III ANALISIS MASALAH
% Pembagian subbab tidak rigid dan dapat bervariasi. Bab ini minimal berisi analisis kebutuhan
% fungsional dan nonfungsional, analisis berbagai alternatif solusi yang dapat ditawarkan, dan
% metode pemilihan solusi yang diusulkan.
% ============================================================================================
\chapter{ANALISIS MASALAH}
\label{chap:analisis-masalah}
\section{Analisis Kondisi Saat Ini}
Tahap ini bertujuan untuk memahami kondisi aktual terkait alur kedatangan tamu di Gedung ITB Innovation Park serta kendala yang muncul dalam proses operasionalnya.
Berdasarkan hasil observasi lapangan dan wawancara dengan petugas keamanan dan resepsionis, dapat disimpulkan bahwa sistem kontrol akses untuk tamu saat ini masih sepenuhnya manual dan belum terintegrasi dengan sistem manajemen gedung.
\subsection{Alur Masuk Tamu Saat Ini}
Proses penerimaan tamu saat ini mengandalkan interaksi langsung dengan petugas resepsionis atau petugas keamanan. Alur yang berlangsung di lapangan adalah sebagai berikut:
\begin{enumerate}
    \item Tamu memasuki lobi dan langsung berinteraksi dengan petugas keamanan. Petugas menanyakan keperluan tamu, identitas, serta pihak yang ingin ditemui. Proses verifikasi dilakukan secara verbal tanpa pencocokan dokumen atau identitas digital apa pun
    \item Tidak ada sistem registrasi atau pencatatan tamu yang terpusat. Petugas keamanan hanya melakukan konfirmasi secara manual kepada tenant atau staf terkait melalui pesan singkat atau panggilan telepon untuk memastikan bahwa tamu tersebut memang memiliki janji atau terdapat kebutuhan kunjungan.
    \item Tamu diperbolehkan melanjutkan ke area lift tanpa mekanisme autentikasi tambahan. Seluruh tamu dapat mengakses area lift setelah mendapatkan persetujuan verbal dari petugas keamanan.
    \item Tidak ada pencatatan waktu masuk dan keluar. Aktivitas tamu tidak terdokumentasi, sehingga tidak ada data historis yang dapat digunakan untuk kebutuhan audit keamanan, monitoring, atau analisis operasional.
\end{enumerate}

Alur ini menunjukkan bahwa seluruh proses penerimaan tamu sangat bergantung pada komunikasi informal dan pengawasan manual oleh resepsionis. Ketergantungan pada proses manual ini menimbulkan sejumlah risiko, antara lain keterbatasan akurasi verifikasi identitas, potensi akses tanpa izin saat petugas lengah, serta tidak tersedianya data kunjungan untuk evaluasi keamanan gedung.
\subsection{Kondisi Lobi Saat Ini}
Hasil peninjauan area lobi menunjukkan bahwa gedung belum dilengkapi dengan perangkat pendukung sistem kontrol akses apa pun.
Akses menuju lift terbuka secara langsung dari area lobi, hanya diawasi oleh petugas keamanan.
Tidak adanya sistem kontrol akses mengakibatkan seluruh proses penyaringan tamu harus dilakukan secara manual dan tidak memiliki \textit{backup} apabila petugas tidak berada di tempat. 
Gambar \ref{gambar:lobi-iip} menunjukkan gambaran dari lobi Gedung ITB Innovation Park saat ini.

\begin{figure}[H] % pilihan opsi yang disarankan: t = top, b = bottom, h = here
	\centering
  \captionsetup{justification=centering}
    	\includegraphics[width=0.7\textwidth]{image/denah-sebelum.png}
	\caption{Kondisi lapangan dari lobi Gedung IIP}
	\label{gambar:lobi-iip}
\end{figure}

\section{Analisis Kebutuhan}
Tahap ini bertujuan untuk mengidentifikasi kebutuhan yang harus dipenuhi dalam pengembangan sistem kontrol akses di Gedung ITB Innovation Park. Analisis dilakukan dengan memperhatikan kondisi operasional saat ini, pola interaksi antara tamu dan petugas, serta kebutuhan pengelola gedung terhadap keamanan dan efisiensi. Hasil analisis kebutuhan ini menjadi dasar dalam menentukan fitur, komponen, dan alur sistem yang akan dikembangkan.
\subsection{Identifikasi Masalah Pengguna}
Berdasarkan hasil observasi di lobi Gedung ITB Innovation Park, proses pengaturan tamu yang berkunjung menunjukkan sejumlah permasalahan yang memengaruhi efisiensi dan keamanan operasional. Tamu yang datang belum memiliki alur registrasi yang jelas dan terstruktur, sehingga verifikasi identitas sangat bergantung pada percakapan langsung dengan resepsionis. Kondisi ini menyebabkan alur masuk yang tidak konsisten dan rentan menimbulkan antrean pada jam sibuk. Selain itu, ketiadaan sistem pencatatan kunjungan membuat aktivitas tamu tidak terdokumentasi, sehingga menyulitkan pengelola gedung dalam melakukan penelusuran dan audit keamanan. Pengguna gedung, baik resepsionis maupun pengelola, juga tidak memiliki alat bantu untuk mengontrol area mana saja yang boleh diakses tamu maupun jangka waktu kunjungannya. Permasalahan tersebut menunjukkan perlunya sistem yang dapat mengatur proses kunjungan tamu secara lebih tertib, aman, dan terdigitalisasi.
\subsection{Kebutuhan Fungsional}
Kebutuhan fungsional merupakan fitur atau fungsi utama yang harus dimiliki oleh sistem agar dapat memenuhi kebutuhan pengguna dan menyelesaikan permasalahan yang telah diidentifikasi pada subbab sebelumnya.
Rincian kebutuhan nonfungsional disajikan pada tabel \ref{tbl:func-req}.
\begin{table}[H]
\centering
\begin{tabularx}{\textwidth}{|p{3.5cm}|X|}
  \hline
  \textbf{Nama Kebutuhan}       & \textbf{Penjelasan} \\
  \hline
  KF-01: Kontrol Akses                 & Sistem harus mampu mengoperasikan pembukaan dan penutupan gerbang fisik secara otomatis berdasarkan hasil validasi, sehingga akses ke area gedung dapat dibatasi sesuai izin yang diberikan. \\
  \hline
  KF-02: Pendeteksi Wajah              & Sistem harus mampu \\
  \hline
  KF-03: Pengenalan Wajah              & Sistem harus dapat mengenali dan memberikan akses kepada pengguna yang memiliki hak akses memasuki gedung. \\
  \hline
  KF-04: Manajemen Data                & Pengguna sistem harus dapat mengakses sistem untuk menambahkan, mengubah atau menghapus data mereka yang digunakan dalam sistem.\\
  \hline
  KF-05: Integrasi Sistem              & Sistem harus dapat melakukan sinkronasi data yang mereka miliki dengan Sistem Manajemen Gedung, dengan tetap memperhatikan regulasi yang berlaku.\\
  \hline
  KF-06: Keselamatan/\textit{safety}   & Sistem wajib memiliki fitur \textit{fail-safe} yang menjamin gerbang terbuka secara otomatis pada situasi darurat atau saat terjadi pemadaman daya. \\
  \hline
\end{tabularx}
\caption{Kebutuhan Fungsional Sistem}
\label{tbl:func-req}
\end{table}
\subsection{Kebutuhan Nonfungsional}
Kebutuhan nonfungsional menentukan batasan kualitas yang harus dipenuhi sistem, seperti target akurasi, kapasitas basis data, waktu respons, keamanan data, serta keandalan operasi.
Rincian kebutuhan nonfungsional ditampilkan pada tabel \ref{tbl:nonfunc-req}.
\input{table/nonfungsionalReq.tex}

\section{Analisis Pemilihan Solusi}
Setelah kebutuhan sistem ditetapkan, tahap berikutnya adalah mengevaluasi berbagai pilihan solusi yang dapat memenuhi kebutuhan tersebut.
Setiap alternatif akan dianalisis dan dibandingkan melalui pendekatan \textit{trade-off} untuk menentukan opsi yang paling sesuai dengan tujuan pengembangan sistem.
\subsection{Alternatif Solusi}
Berikut merupakan rangkuman alternatif solusi yang dapat digunakan untuk memenuhi masing-masing kebutuhan fungsional dari sistem.
\begin{enumerate}
    \item KF-1 Kontrol Akses
    \begin{enumerate}[a.]
        \item \textit{Swing Barrier}, yaitu tipe gerbang yang membuka dan menutup ke arah dalam atau luar.
        \item \textit{Flap Barrier}, yaitu tipe gerbang yang membuka dan menutup dengan menggeser penghalang ke arah samping.
        \item \textit{Tripod Gate}, yaitu tipe gerbang dengan 3 batang besi yang dapat berputar searah saat kunci terbuka.
    \end{enumerate}

    \item KF-2 Deteksi Wajah
    \begin{enumerate}[a.]
        \item RetinaFace, merupakan detektor wajah satu tahap dengan akurasi tinggi yang mampu mendeteksi wajah dalam berbagai kondisi pose dan pencahayaan. Model ini memanfaatkan landmark detection untuk meningkatkan stabilitas proses alignment pada tahap selanjutnya. 
        \item YOLO-face adalah pendekatan deteksi wajah berbasis arsitektur YOLO yang berorientasi pada kecepatan dan kemampuan real-time. Metode ini memperlakukan wajah sebagai objek yang dideteksi melalui bounding box secara langsung, sehingga cocok untuk implementasi sistem kontrol akses yang memerlukan respons cepat. Model YOLO-based face detector telah banyak digunakan dalam sistem deteksi wajah pada perangkat edge dan aplikasi absensi otomatis.
        \item Blazeface adalah detektor wajah yang dirancang untuk efisiensi tinggi pada perangkat mobile dengan sumber daya terbatas. Dengan ukuran model yang kecil dan latensi sangat rendah, BlazeFace dapat digunakan untuk prototipe sistem atau perangkat kontrol akses dengan komputasi ringan tanpa mengorbankan performa secara signifikan.
    \end{enumerate}

    \item KF-03 Pengenalan Wajah
    
    Alternatif Solusi untuk pengenalan wajah adalah kombinasi dari alternatif solusi untuk \textit{Feature Extraction} dan \textit{Matching} 
    \textit{Feature Extraction}:
    \begin{enumerate}[a.]
        \item ResNet-50, yaitu arsitektur CNN dengan 50 layer yang menggunakan mekanisme residual learning, menjadi backbone umum untuk pengenalan wajah karena mampu mengekstraksi fitur wajah yang kuat dan stabil.
        \item MobileFaceNet, yaitu model CNN ringan yang cocok digunakan pada perangkat kecil, tetapi masih cukup baik dalam menangkap ciri penting dari wajah.
    \end{enumerate}
    \medskip
    \medskip
    \textit{Loss Function}:
    \begin{enumerate}[a.]
        \item ArcFace, metode pelatihan yang membedakan wajah dengan `menjauhkan' jarak antar wajah yang berbeda dan `mendekatkan' wajah dari orang yang sama. 
        \item MagFace, metode pelatihan yang tidak hanya membuat model mengenali wajah, tapi juga bisa menilai kualitas foto wajah, sehingga hasil pengenalannya lebih stabil.
    \end{enumerate}

    \item KF-4 Manajemen Pendaftaran
    \begin{enumerate}[a.]
        \item Aplikasi Web, menggunakan website yang dapat diakses melalui browser untuk pendaftaran.
        \item Aplikasi Desktop, menggunakan aplikasi berbasis desktop untuk perangkat PC resepsionis.
        \item Aplikasi Mobile, menggunakan aplikasi berbasis mobile untuk ponsel pengguna.
    \end{enumerate}

    \item KF-5 Integrasi Sistem
    \begin{enumerate}[a.]
        \item Integrasi berbasis API, memungkinkan sistem saling bertukar data secara langsung melalui HTTP \textit{request} dan \textit{response}.
        \item Integrasi berbasis \textit{Message Queue}, mengirim dan memproses pesan secara asinkron melalui antrian.
        \item Integrasi berbasis \textit{Webhook}, mengirimkan notifikasi otomatis ke sistem lain setiap adanya peristiwa (\textit{event}) tertentu.
    \end{enumerate}

    \item KF-6 Keselamatan/\textit{safety}
    \begin{enumerate}[a.]
        \item Gerbang \textit{fail-safe}, yaitu gerbang yang memiliki kondisi terbuka saat tidak mendapatkan aliran listrik.
        \item Tombol Darurat, yaitu tombol yang dapat membuka gerbang tanpa autentikasi.
        \item Gerbang \textit{fail-safe} + Tombol Darurat, yaitu penggabungan solusi yang memungkinkan gerbang terbuka saat listrik padam atau tombol ditekan.
    \end{enumerate}

\end{enumerate}
\subsection{Analisis Penentuan Solusi}
Untuk menentukan pemilihan solusi terbaik, dilakukan analisis kuantitatif untuk memilih solusi terbaik. Analisis kuantitatif dilakukan dengan metode \textit{Weighted Scoring Model} (WSM) untuk membandingkan setiap alternatif solusi. berikut merupakan analisis WSM dari setiap alternatif solusi setiap kebutuhan.
% \input{table/tab_alternatif_solusi.tex}
\subsubsection{KF-1: Kontrol Akses}
Berikut merupakan kriteria penilaian dari fungsionalitas kontrol akses.

\input{table/wsm_fr1_criteria.tex}
Berdasarkan kriteria tersebut, berikut merupakan \textit{Weighted Scoring Model} dari setiap alternatif solusi.

\begin{table}[H]
\centering
\caption{Analisis Penentuan Solusi Perangkat Gerbang Menggunakan Weighted Scoring Model}
\label{tab:wsm-perangkat-gerbang}
\small
\setlength{\tabcolsep}{6pt}

\begin{tabular}{l c c c c}
\toprule
\textbf{Kriteria Penilaian} 
& \textbf{Bobot} 
& \textbf{Swing Barrier} 
& \textbf{Flap Barrier} 
& \textbf{Tripod} \\ 
\midrule

Keamanan Fisik 
& 35\% 
& 4 
& 3 
& 2 \\

Kecepatan Melalui Gerbang (Throughput)
& 25\% 
& 4 
& 5 
& 3 \\

Kenyamanan \& Aksesibilitas
& 15\% 
& 5 
& 3 
& 2 \\

Efisiensi Energi \& Perawatan
& 15\% 
& 3 
& 4 
& 4 \\

Keandalan Operasional
& 10\%
& 4
& 3
& 3 \\

\midrule
\textbf{Total Skor}
& \textbf{100\%}
& \textbf{4.00}
& \textbf{3.65}
& \textbf{2.65} \\
\bottomrule
\end{tabular}
\end{table}

Analisis menunjukkan bahwa \textit{swing barrier} merupakan solusi terbaik yang menawarkan keseimbangan antara keamanan, kecepatan, dan aksesibilitas yang baik dibandingkan dengan alternatif solusi lainnya.

\subsubsection{KF-2: Deteksi Wajah}
Berikut merupakan kriteria penilaian dari fungsionalitas deteksi wajah.
\input{table/wsm_fr2_criteria.tex}
Berdasarkan kriteria tersebut, berikut merupakan \textit{Weighted Scoring Model} dari setiap alternatif solusi.
\input{table/wsm_fr2.tex}
Analisis menunjukkan bahwa YOLO-Face terpilih karena menawarkan performa paling seimbang untuk implementasi di Raspberry Pi.

\subsubsection{KF-3: Pengenalan Wajah}
Berikut merupakan kriteria penilaian dari fungsionalitas pengenalan wajah.
\input{table/wsm_fr3_criteria.tex}
Berdasarkan kriteria tersebut, berikut merupakan\textit{Weighted Scoring Model} dari setiap alternatif solusi berikut : 
\begin{enumerate}
    \item Solusi 1: ResNet-50 + ArcFace
    \item Solusi 2: ResNet-50 + MagFace
    \item Solusi 3: MobileFaceNet + ArcFace
    \item Solusi 4: MobileFaceNet + MagFace  
\end{enumerate}
\input{table/wsm_fr3.tex} 
Berdasarkan analisis, terlihat bahwa solusi terbaik adalah MobileFaceNet + MagFace, dimana kombinasi ini menawarkan solusi pengenalan wajah yang ringan secara komputasi namun tetap tangguh terhadap citra gambar yang bervariatif, sehingga sangat cocok digunakan pada perangkat terbatas seperti kontroler Raspberry Pi. MobileFaceNet memberikan keunggulan sebagai model yang sangat efisien dengan jumlah parameter yang rendah, memungkinkan inferensi cepat dan akurasi yang memadai. Sementara itu, MagFace merupakan \textit{loss function} yang peka terhadap kualitas citra gambar, sehingga sistem dapat mengurangi kesalahan pada saat mendeteksi dalam pencahayaan rendah, pose yang bervariasi, atau gangguan (\textit{noise}).

\subsubsection{KF-4: Manajemen Pendaftaran}
Berikut merupakan kriteria penilaian dari fungsionalitas pengenalan wajah.
\begin{table}[H]
\centering
\caption{Kriteria Evaluasi Proses Pendaftaran Mandiri}
\label{tab:kriteria-pendaftaran}
\begin{tabularx}{\textwidth}{|p{3.5cm}|p{2cm}|X|}
\hline
\textbf{Kriteria} & \textbf{Bobot(\%)} & \textbf{Alasan} \\
\hline
Efisiensi Waktu & 35\% & 
Proses pendaftaran yang cepat mengurangi usaha dan waktu pengguna, sesuai dengan heuristic \textit{Efficiency of Use / Flexibility and Efficiency} \autocite{nielsen1994usability}. \\
\hline
Kemudahan Penggunaan & 30\% & 
Antarmuka yang intuitif dan rendah beban kognitif memungkinkan pengguna menyelesaikan pendaftaran tanpa kebingungan, sejalan dengan heuristics \textit{Match between system and the real world} dan \textit{Recognition rather than recall} \autocite{nielsen1994usability}. \\
\hline
Kompatibilitas Perangkat & 20\% & 
Sistem harus konsisten dan berjalan pada berbagai perangkat agar akses lebih luas, sesuai dengan heuristik \textit{Consistency and Standards} \autocite{nielsen1994usability}. \\
\hline
Persepsi Privasi & 15\% & 
Pengguna merasa aman ketika data pribadi tidak disalahgunakan, sesuai dengan prinsip keamanan dalam \textit{usability} \autocite{nielsen1994usability}. \\
\hline
\end{tabularx}
\end{table}
Berdasarkan kriteria tersebut, berikut merupakan \textit{Weighted Scoring Model} dari setiap alternatif solusi.
\input{table/wsm_fr4.tex} 
Berdasarkan analisis, terlihat bahwa Aplikasi \textit{web} menawarkan efisiensi waktu, kemudahan, kompatibilitas, dan privasi yang lebih baik dibandingkan dengan \textit{mobile} dan \textit{dekstop}.
\subsubsection{KF-5: Integrasi Sistem}
Berikut merupakan kriteria penilaian dari fungsionalitas integrasi sistem.
\begin{table}[H]
\centering
\caption{Kriteria Penilaian Metode Integrasi Data}
\label{tab:kriteria}
\begin{tabularx}{\textwidth}{|p{3.5cm}|p{2cm}|X|}
\hline
\textbf{Kriteria} & \textbf{Bobot(\%)} & \textbf{Alasan} \\
\hline

Standarisasi & 40\% & 
Metode integrasi data yang memiliki tingkat standarisasi tinggi umumnya didukung oleh dokumentasi lengkap dan ekosistem pustaka pemrograman yang matang, sehingga memudahkan implementasi dan interoperabilitas antarsistem. \\ 
\hline

Kemudahan Debug & 30\% & 
Kemudahan dalam melakukan pelacakan dan penanganan kesalahan (\textit{debugging}) penting untuk mempercepat proses pengembangan dan mengurangi risiko kegagalan integrasi saat sistem berjalan. \\ 
\hline

Real-time & 30\% & 
Kemampuan sinkronisasi data secara real-time memastikan informasi antar sistem selalu mutakhir, sehingga mendukung proses operasional yang memerlukan respons cepat dan konsistensi data. \\ 
\hline

\end{tabularx}
\end{table}

Berdasarkan kriteria tersebut, berikut merupakan \textit{Weighted Scoring Model} dari setiap alternatif solusi.
\begin{table}[H]
\centering
\caption{Analisis Penentuan Solusi Integrasi Sistem Menggunakan Weighted Scoring Model}
\label{tab:wsm-kf5}
\small
\setlength{\tabcolsep}{4pt}

\begin{tabular}{l c c c c}
\toprule
\textbf{Kriteria Penilaian} 
& \textbf{Bobot} 
& \textbf{REST API} 
& \textbf{Webhook} 
& \textbf{Manual} \\ 
\midrule

Standarisasi 
& 40\% 
& 5 (2.0) 
& 4 (1.6) 
& 1 (0.4) \\

Kemudahan Debug 
& 30\% 
& 5 (1.5) 
& 3 (0.9) 
& 2 (0.6) \\

Real-time 
& 30\% 
& 3 (0.9) 
& 5 (1.5) 
& 1 (0.3) \\

\midrule
\textbf{Total Skor} 
& \textbf{100\%} 
& \textbf{4.4} 
& \textbf{4.0} 
& \textbf{1.3} \\

\bottomrule
\end{tabular}
\end{table}
 
Analisis penilaian menunjukkan bahwa REST API merupakan solusi terbaik dibandingkan alternatif solusi lainnya, dimana REST API lebih unggul dalam standarisasi kemudahan dalam melakukan \textit{debugging}. 

\subsubsection{KF-6: Keselamatan/safety}
Berikut merupakan kriteria penilaian dari fungsionalitas integrasi sistem.
\begin{table}[H]
\centering
\caption{Kriteria Penilaian KF-6: Mekanisme Keselamatan}
\label{tab:kf6_kriteria}
\begin{tabularx}{\textwidth}{|p{3.5cm}|p{2cm}|X|}
\hline
\textbf{Kriteria} & \textbf{Bobot(\%)} & \textbf{Alasan} \\
\hline

Keandalan & 50\% & 
Sistem keselamatan harus tetap berfungsi dalam kondisi kritis seperti kebakaran atau pemadaman listrik untuk menjamin proses evakuasi berjalan tanpa hambatan. \\ 
\hline

Kepatuhan & 30\% & 
Pemenuhan standar keselamatan bangunan dan evakuasi menjadi syarat utama agar mekanisme keselamatan sesuai regulasi yang berlaku. \\ 
\hline

Kecepatan Respon & 20\% & 
Respons cepat dalam membuka kunci setelah sinyal darurat diterima memastikan tidak ada keterlambatan pada proses evakuasi. \\ 
\hline

\end{tabularx}
\end{table}

Berdasarkan kriteria tersebut, berikut merupakan \textit{Weighted Scoring Model} dari setiap alternatif solusi.
\begin{table}[H]
\centering
\caption{Analisis Penentuan Mekanisme Keselamatan Menggunakan Weighted Scoring Model}
\label{tab:wsm-kf6}
\small
\setlength{\tabcolsep}{6pt}

\begin{tabular}{l c c c c}
\toprule
\textbf{Kriteria Penilaian} 
& \textbf{Bobot} 
& \textbf{Otomatis (Fail-Safe)} 
& \textbf{Manual} 
& \textbf{Kombinasi} \\ 
\midrule

Keandalan 
& 50\% 
& 4 (2.00) 
& 3 (1.50)
& 5 (2.50)\\

Kepatuhan 
& 30\% 
& 5 (1.50) 
& 2 (0.60)
& 5 (1.50)\\

Kecepatan Respon 
& 20\% 
& 5 (1.00)
& 2 (0.40)
& 5 (1.00)\\

\midrule
\textbf{Total Skor} 
& \textbf{100\%} 
& \textbf{4.5} 
& \textbf{2.5} 
& \textbf{5.0} \\

\bottomrule
\end{tabular}
\end{table}
 
Solusi kombinasi, yaitu gabungan dengan solusi otomatis(\textit{fail-safe}) dan tombol daruat merupakan solusi terbaik pada sistem. Solusi ini memastikan aspek keselamatan yang mumpuni dengan keandalan, kepatuhan dan kecepatan respon yang tinggi.
