% ==========================================
% BAB IV DESAIN KONSEP SOLUSI
% ==========================================
\chapter{DESAIN KONSEP SOLUSI}
\label{chap:desain-konsep-solusi}
Tujuan dari bab desain konsep solusi adalah menjelaskan model konseptual dan uraian desain yang telah dipilih pada bab sebelumnya terkait penerapan sistem pengenalan wajah untuk kontrol akses di lobi ITB Innovation Park.
\section{Diagram Konseptual}
Pada tahap ini dijelaskan gambaran konseptual mengenai perubahan alur kontrol akses yang terjadi setelah sistem baru diimplementasikan. Model konseptual ini bertujuan menunjukkan bagaimana proses operasional di lobi Gedung ITB Innovation Park (IIP) akan berubah dibandingkan dengan mekanisme sebelumnya. Penyajian diagram dilakukan untuk memudahkan pembaca memahami perbedaan aliran proses before dan after pemasangan gerbang otomatis berbasis pengenalan wajah serta metode autentikasi alternatif.

Sebelum sistem dikembangkan, seluruh proses verifikasi identitas pengunjung, baik karyawan maupun tamu masih sepenuhnya dilakukan oleh petugas keamanan secara manual. Gambar \ref{gambar:alur-before} menampilkan alur kontrol akses lama, mulai dari kedatangan pengguna hingga diperbolehkan memasuki area lift.
\begin{figure}[H] % pilihan opsi yang disarankan: t = top, b = bottom, h = here
	\centering
  \captionsetup{justification=centering}
    	\includegraphics[width=0.7\textwidth]{image/alur-before.png}
	\caption{Alur kontrol akses sebelum penerapan sistem gerbang}
	\label{gambar:alur-before}
\end{figure}

Setelah sistem dirancang dan dipilih pada bab sebelumnya, alur kontrol akses diperbarui menjadi lebih terstruktur dan otomatis.
Untuk karyawan, proses autentikasi akan dilakukan menggunakan \textit{face recognition} sebagai metode utama. 
Alur tersebut ditampilkan pada Gambar \ref{gambar:alur-after-karyawan}.
Sementara itu, tamu memiliki dua pilihan autentikasi, yaitu menggunakan pengenalan wajah atau kartu RFID hasil verifikasi identitas di resepsionis.
Ilustrasi alur ini ditampilkan pada Gambar \ref{gambar:alur-after-tamu}.
\begin{figure}[H] % pilihan opsi yang disarankan: t = top, b = bottom, h = here
	\centering
  \captionsetup{justification=centering}
    	\includegraphics[width=0.7\textwidth]{image/alur-after-karyawan.png}
	\caption{Alur kontrol akses sesudah penerapan sistem gerbang untuk karyawan gedung}
	\label{gambar:alur-after-karyawan}
\end{figure}
\begin{figure}[H] % pilihan opsi yang disarankan: t = top, b = bottom, h = here
	\centering
  \captionsetup{justification=centering}
    	\includegraphics[width=0.7\textwidth]{image/alur-after-tamu.png}
	\caption{Alur kontrol akses sesudah penerapan sistem gerbang untuk tamu gedung}
	\label{gambar:alur-after-tamu}
\end{figure}

Selain kedua alur tersebut, sistem juga dirancang untuk berfungsi secara aman pada kondisi darurat, misalnya saat terjadi kebakaran atau pemadaman listrik.
Dalam skenario tersebut, gerbang akan masuk ke mode \textit{fail-safe} dan terbuka otomatis.
Alur untuk kontrol akses saat kondisi darurat, ditampilkan pada Gambar \ref{gambar:alur-after-darurat}.
\begin{figure}[H] % pilihan opsi yang disarankan: t = top, b = bottom, h = here
	\centering
  \captionsetup{justification=centering}
    	\includegraphics[width=0.7\textwidth]{image/alur-after-darurat.png}
	\caption{Alur kontrol akses sesudah penerapan sistem gerbang saat kondisi darurat}
	\label{gambar:alur-after-darurat}
\end{figure}

Selain menggambarkan perubahan alur proses, model konseptual juga mencakup visualisasi tata letak lobi gedung sebelum dan sesudah pemasangan gerbang.
Denah awal ditampilkan pada Gambar \ref{gambar:denah-sebelum}, sedangkan penempatan gerbang baru, yang terdiri dari tiga gerbang ditampilkan pada Gambar \ref{gambar:denah-sesudah-lurus}.
\begin{figure}[H] % pilihan opsi yang disarankan: t = top, b = bottom, h = here
	\centering
  \captionsetup{justification=centering}
    	\includegraphics[width=0.7\textwidth]{image/denah-sebelum.png}
	\caption{Denah lobi sebelum pemasangan sistem gerbang}
	\label{gambar:denah-sebelum}
\end{figure}
\begin{figure}[H] % pilihan opsi yang disarankan: t = top, b = bottom, h = here
	\centering
  \captionsetup{justification=centering}
    	\includegraphics[width=0.7\textwidth]{image/denah-sesudah-lurus.png}
	\caption{Denah lobi setelah pemasangan sistem gerbang}
	\label{gambar:denah-sesudah-lurus}
\end{figure}

\section{Penjelasan Desain}
Bagian ini akan menjelaskan secara ringkas bagaimana rancangan sistem kontrol akses akan diimplementasikan. Penjelasan desain ini meliputi keterhubungan antarkomponen, penjelasan tentang komponen yang dipilih secara ringkas, logika proses autentikasi serta logika proses pendaftaran.
\subsection{Spesifikasi Perangkat Keras}
Gerbang bertipe \textit{swing barrier} dipilih untuk mendukung aksesibilitas yang luas, sesuai dengan standar keselamatan \parencite{simarmata2021gerbang}. Dalam implementasinya, Sistem dirancang akan memiliki tiga gerbang normal pada satu sisi dan satu gerbang disabiilitas untuk sisi lainnya, dengan gerbang disabilitas akan memiliki jalur minimal yang lebih lebar (80 cm) dibandingkan dengan gerbang normal (60 cm). Gambar \ref{gambar:dimensi-gerbang} menunjukkan dimensi dari gerbang yang akan digunakan.
\begin{figure}[H] % pilihan opsi yang disarankan: t = top, b = bottom, h = here
	\centering
  \captionsetup{justification=centering}
    	\includegraphics[width=1\textwidth]{image/gerbang-satuan.png}
	\caption{Dimensi dari \textit{swing barrier.}}
	\label{gambar:dimensi-gerbang}
\end{figure}

Gerbang yang akan digunakan kemudian akan terhubung langsung dengan sistem pengenalan wajah dan RFID, Berdasarkan analisis kebutuhan pada Bab III, spesifikasi perangkat keras yang dipilih meliputi:
\begin{enumerate}
    \item \textbf{Unit Pemrosesan:} Raspberry Pi 4 Model B (4GB) dipilih karena kemampuan \textit{edge computing} yang memadai untuk menjalankan algoritma \textit{Deep Learning} \parencite{raspberrypi}.
    \item \textbf{Visual:} Raspberry Pi Camera Module v3 dengan fitur HDR untuk mengatasi kondisi pencahayaan lobi.
    \item \textbf{Antarmuka:} Layar LCD 5 inci HDMI untuk menampilkan status akses kepada pengguna.
    \item \textbf{Autentikasi Sekunder:} Modul RFID \textit{Reader} sebagai opsi akses cadangan.
    \item \textbf{Kontrol Akses:} Modul Relay 5V untuk memicu pembukaan gerbang melalui mekanisme kontak kering (\textit{dry contact}) \parencite{kainz2019raspberry}.
\end{enumerate}
Mekanisme fisik gerbang menggunakan \textit{Swing Barrier} untuk mendukung aksesibilitas yang luas, sesuai dengan standar keselamatan \parencite{simarmata2021gerbang}.
\subsection{Diagram Komponen}
Rancangan sistem kontrol akses ini akan terdiri dari beberapa komponen yang saling terhubung. Gambar \ref{gambar:hubungan-komponen} Menunjukkan diagram komponen dari sistem.
\begin{figure}[H] % pilihan opsi yang disarankan: t = top, b = bottom, h = here
	\centering
  \captionsetup{justification=centering}
    	\includegraphics[width=0.7\textwidth]{image/hubungan-komponen.png}
	\caption{Diagram komponen dari rancangan sistem}
	\label{gambar:hubungan-komponen}
\end{figure}
Berdasarkan gambar, terlihat bahwa sistem kontrol akses dengan gerbang otomatis terdiri dari komponen yang ada pada pemilihan solusi yaitu gerbang, sistem pengenalan wajah (kamera, kontroler, \textit{display}), sistem RFID (RFID \textit{Reader}, Kontroler), Tombol darurat, serta komponen-komponen seperti sumber listrik untuk memastikan sistem bekerja sesuai dengan kebutuhan.

\subsection{Logika Autentikasi}
Logika autentikasi menjelaskan proses pengenalan pengguna yang dilakukan oleh sistem pengenalan wajah. Gambar \ref{gambar:logika-autentikasi} adalah alur proses dari proses pengenalan wajah untuk autentikasi pengguna. 

Setiap beberapa waktu, sistem akan mengambil \textit{frame} video melalui kamera, jika mendeteksi adanya wajah pada frame, sistem kemudian akan melakukan proses ekstrasi dari fitur wajah menjadi sebuah vektor. hasil ini kemudian dibandingkan dengan data tersimpan untuk mencari vektor terdekat. Jika tidak ditemukan, sistem akan menunjukkan bahwa akses masuk ditolak.

\begin{figure}[H] % pilihan opsi yang disarankan: t = top, b = bottom, h = here
	\centering
  \captionsetup{justification=centering}
    	\includegraphics[width=0.6\textwidth]{image/proses-autentikasi.png}
	\caption{Logika autentikasi dari rancangan sistem}
	\label{gambar:logika-autentikasi}
\end{figure}

Jika menemukan jarak vektor yang dibawah \textit{treshold}, sistem akan menampilkan akses diterima. Setelah itu, sistem akan mengambil data pengguna yang dikenali tersebut, lalu mencatat log akses dan membuka gerbang. Sistem kemudian akan menunggu beberapa saat sebelum akhirnya kembali mengambil \textit{frame} video untuk mendeteksi pengguna berikutnya.

\subsection{Alur Pendaftaran pada Sistem Kontrol Akses}
Sistem pendaftaran digunakan oleh pengguna untuk mendaftarkan wajah mereka, sehingga pengguna dapat dikenali dan diberikan akses masuk. Berikut merupakan alur pendaftaran dari sistem kontrol akses yang akan dikembangkan.
\begin{figure}[H] % pilihan opsi yang disarankan: t = top, b = bottom, h = here
	\centering
  \captionsetup{justification=centering}
    	\includegraphics[width=0.6\textwidth]{image/alur-pendaftaran.jpg}
	\caption{Alur pendaftaran pada sistem kontrol akses}
	\label{gambar:alur-pendaftaran}
\end{figure}
Pendaftaran dimulai dengan memasukkan data pengguna berupa nama, kemudian sistem akan meminta akses kamera pada perangkat. Setelah pengguna mengambil citra gambar, sistem kemudian akan melakukan proses deteksi wajah, dan jika terdeteksi, wajah akan disimpan ke database. Proses ini berlangsung sebanyak tiga kali untuk meningkatkan akurasi pengenalan wajah. 