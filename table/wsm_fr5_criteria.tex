\begin{table}[H]
\centering
\caption{Kriteria Penilaian Metode Integrasi Data}
\label{tab:kriteria}
\begin{tabularx}{\textwidth}{|p{3.5cm}|p{2cm}|X|}
\hline
\textbf{Kriteria} & \textbf{Bobot(\%)} & \textbf{Alasan} \\
\hline

Standarisasi & 40\% & 
Metode integrasi data yang memiliki tingkat standarisasi tinggi umumnya didukung oleh dokumentasi lengkap dan ekosistem pustaka pemrograman yang matang, sehingga memudahkan implementasi dan interoperabilitas antarsistem. \\ 
\hline

Kemudahan Debug & 30\% & 
Kemudahan dalam melakukan pelacakan dan penanganan kesalahan (\textit{debugging}) penting untuk mempercepat proses pengembangan dan mengurangi risiko kegagalan integrasi saat sistem berjalan. \\ 
\hline

Real-time & 30\% & 
Kemampuan sinkronisasi data secara real-time memastikan informasi antar sistem selalu mutakhir, sehingga mendukung proses operasional yang memerlukan respons cepat dan konsistensi data. \\ 
\hline

\end{tabularx}
\end{table}
